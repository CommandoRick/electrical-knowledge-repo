% Header and Footer
\pagestyle{fancy}
%\fancyfoot[L]{}
\fancyfoot[C]{By: \myName}
\fancyfoot[R]{\thepage}
\renewcommand{\headrulewidth}{0.4pt}
\renewcommand{\footrulewidth}{2pt}

\section{UQ Subjects} %\label{sec_introduction}
This chapter goes through the UQ courses that was undertaken from 2016 to 2019. The format will be as follows, for each section, where possible:
\begin{enumerate}
  \item Lecture notes (Use ``LEC\#\#: TITLE HERE'' for each heading)
  \item Tutorial questions (Use ``TUT\#\#: TITLE HERE'' for each heading)
  \item Summary of all equations used (Use ``EQU\#\#: TITLE HERE'' for each heading)
  \item References \& other helping material (Use ``REF\#\#: TITLE HERE'' for each heading)
  \item Australian Standards (Use ``STD\#\#: TITLE HERE'' for each heading)
\end{enumerate}

In terms of text colour and highlights, the format will be as follows where possible:
\begin{enumerate}
  \item Black = normal text
  \item \textcolor{red}{Red = Important}
  \item \textcolor{blue}{Blue = References}
  \item \textcolor{green}{Green = Key Takeaways}
\end{enumerate}

% TOPIC: CODING
\subsection{CSSE2002 - Java Language}
\clearpage

\subsection{CSSE2010 - Embedded Programming}
\clearpage

\subsection{CSSE2310 - C Language}
\clearpage

\subsection{CSSE3010 - Advanced Embedded}
\clearpage

% TOPIC: MATHEMATICS
\subsection{MATH1051 - Linear Calculus}
\clearpage

\subsection{MATH2001 - Advanced Calculus}
\clearpage

\subsection{MATH2010 - Partial Differential Equations}
\clearpage

\subsection{STAT2202 - Advanced Statistics}
\clearpage

% TOPIC: ELECTRICAL KNOWLEDGE
\subsection{ELEC1300 - Electronics Basics}
\clearpage

\subsection{ELEC2003 - Electronics \& Circuits Pt.1}
\clearpage

\subsection{ELEC2004 - Electronics \& Circuits Pt.2}
\subsubsection{LEC01: Capacitors and Inductors, RL and RC Circuits}
\textsc{\large Capacitors}\\
Capacitors and inductors are linear circuit elements that can store electrical energy. The ideal capacitor stores energy in the form of \textbf{charge}.

\begin{align} \label{eq_ELEC2004_capacitance}
  C &= \frac{\epsilon A}{d}
\end{align}
Where:
\begin{itemize}
  \item C = capacitance in Farads ($F$)
  \item A = conductor plates area (both top and bottom) ($mm^2$)
  \item $\epsilon$ = dielectric of permitivity (constant)
  \item d = plate separation distance ($m$)
\end{itemize}

\begin{align} \label{eq_ELEC2004_charge}
  Q &= CV
\end{align}
Where:
\begin{itemize}
  \item Q = stored charge
  \item C = capacitance ($F$)
  \item V = applied voltage ($V$)
\end{itemize}

In DC, a capacitor is effectively an open circuit; when a steady voltage is applied. \textcolor{red}{When the voltage changes, the stored charge changes also as per equation \eqref{eq_ELEC2004_charge} by taking the derivative.} Thus,
\begin{align}
  i(t) &= \frac{dq(t)}{dt} = C \frac{dv(t)}{dt}
\end{align}
\textcolor{green}{KEY TAKEAWAY: Change in voltage induces a current because there are charges moving. This electrical energy is stored in ``capacitance'' in the form of an electric field.}\\
\\
Energy storage in capacitors is calculated by integrating the instantaneous power $P(t)$. Thus,
\begin{align}
  P(t) &= v(t)i(t) = Cv(t)\frac{dv(t)}{dt}
\end{align}
Integrating the instanteous power:
\begin{align}
  W(t) = \frac{1}{2}Cv^2(t)
\end{align}

Capacitors can be combined in series and in parallel to yield a single equivalent capacitance. Note: the behaviour of equivalent capacitance is the opposite of resistors. Series:
\begin{align}
  C_{EQ} &= \frac{1}{\frac{1}{C_1} + \frac{1}{C_2} + \frac{1}{C_3} ...}
\end{align}
Parallel:
\begin{align}
  C_{EQ} &= C_1 + C_2 + C_3 ...
\end{align}

\textsc{\large Inductors}\\
The ideal inductor stores energy in an \textbf{induced magnetic field}.

\begin{align}
  \phi &= LI
\end{align}
Where:
\begin{itemize}
  \item $\phi$ = induced magnetic flux
  \item L = inductance in Henrys ($H$)
  \item I = applied current ($A$)
\end{itemize}

In DC, an inductor is effectively a short circuit (i.e. a wire with no resistance). \textcolor{red}{When the current changes, the induced field also changes.} Thus,

\begin{align}
  v(t) &= \frac{d\phi(t)}{dt} = L \frac{di(t)}{dt}
\end{align}

\textcolor{green}{KEY TAKEAWAY: The rate of change in magnetic flux induces a voltage. Thus, alternating current (AC) induces a change in magnetic field and thus produces a voltage. Inductance is the tendency of an electrical conductor to oppose a change in the electric current flowing through it.}\\
\\
Energy storage in inductors is calculated by integrating the instantaneous power $P(t)$. Thus,
\begin{align}
  P(t) &= v(t)i(t) = Li(t)\frac{di(t)}{dt}
\end{align}
Integrating the instanteous power:
\begin{align}
  W(t) = \frac{1}{2}Li^2(t)
\end{align}

Inductors can be combined in series and in parallel to yield a single equivalent inductance. Note: the behaviour of equivalent inductance is the same as resistors. Series:
\begin{align}
  L_{EQ} &= L_1 + L_2 + L_3 ...
\end{align}
Parallel:
\begin{align}
  L_{EQ} &= \frac{1}{\frac{1}{L_1} + \frac{1}{L_2} + \frac{1}{L_3} ...}
\end{align}

Other topics found:
\begin{itemize}
  \item Solving RC Circuits \& Forced responses \& Transient analysis
  \item Approaches to solve circuits
  \item Class exercises
\end{itemize}
\clearpage

\subsubsection{LEC02: Nodal \& Mesh Analysis and Network Theorems}
\textsc{\large The Basics}\\
These are the fundamental electrical basics:
\begin{itemize}
  \item \textcolor{green}{$V = IR$ (Ohm's Law)}
  \item \textcolor{green}{$P = VI = \frac{V^2}{R} = I^2R$}
  \item \textcolor{green}{Kirchhoff's current law (KCL): Sum of currents into a node = 0}
  \item \textcolor{green}{Kirchhoff's voltage law (KVL): Sum of voltages around a loop = 0}
  \item \textcolor{green}{Series and parallel circuits (voltage and current dividers)}
  \item \textcolor{green}{$R_{series} = R_1 + R_2 + R_3$; $R_{parallel} = \frac{1}{\frac{1}{R_1} + \frac{1}{R_2} + \frac{1}{R_3}}$}
  \item \textcolor{green}{Voltage Divider: $V_{out} = V_{in}\frac{R_x}{R_{total}}$ where $V_{out}$ is the voltage you want; $V_{in}$ is the source voltage; $R_x$ is the resistor you will find $V_{out}$; and $R_{total}$ is the total resistance}
  \item \textcolor{green}{Current Divider: $I_{n} = I_{in}\frac{R_{eq}}{R_{n}}$ where $I_{n}$ is the current you want; $I_{in}$ is the source current; $R_n$ is the resistor in which the desired current flows through $I_{n}$; and $R_{eq}$ is the equivalent resistance of the resistors}
\end{itemize}

\textsc{\large Node and Mesh Analysis}\\
Note: For a network with $n$ nodes, both methods will result in $n$ independent equations, with $n$ unknown variables.
\begin{itemize}
  \item Nodal analysis: Set voltages at each node as the unknown variables, apply KCL. Note: Currents going into the node are positive, and currents going out of the node are negative. \textcolor{red}{Use KCL when the source is a current source; typically.}
  \item Mesh analysis: set the current through each branch as the unknown variables, apply KVL;  Note: choose a direction for a loop and stick to that direction for all loops. Voltages going from negative to positive (e.g. Voltage Sources) are positive and voltages going from positive to negative (e.g. resistors) are impedances. \textcolor{red}{Use KVL when the source is a voltage source; typically.}
\end{itemize}

\textsc{\large Procedure: Nodal (KCL)}
The below steps for the method to use Nodal Analysis:
\begin{enumerate}
  \item Label all circuit parameters - identify the unknown parameters (voltages and currents).
  \item Identify all essential nodes.
  \item Select a node as the reference node (ground node). Assign it a potential of 0 volts. All other voltages are measured with respect to the reference node.
  \item Label the voltages at all other nodes.
  \item Apply KCL at each node and express currents in terms of node voltages.
  \item Solve the resulting equations for node voltages.
  \item WHAT HAPPENS IF THERE ARE V AND I SOURCES?  We may need to define a `super-node' (this is where a voltage source connects two nodes; thus, like a node, Current in = Current out) See Figure \ref{fig_supernode_image}
\end{enumerate}

\begin{figure}
  \includegraphics[width=\textwidth]{/02_UqSubjects/supernode}
  \caption{Snapshot - Super-Nodes in Circuit Analysis (Nodal)}
  \label{fig_supernode_image}
\end{figure}

\textsc{\large Procedure: Mesh (KVL)}\\
The below steps for the method to use Mesh Analysis:
\begin{enumerate}
  \item Label all circuit parameters and identify unknown parameters (voltages and currents).
  \item Identify all meshes of the circuit.
  \item Assign mesh currents and label polarities.
  \item Apply KVL at each mesh and express voltages in terms of the mesh currents.
  \item Solve the resulting simultaneous equations for the mesh currents.
  \item Now that the mesh currents are known, the voltages may be obtained from Ohm's law.
  \item WHAT HAPPENS IF THERE ARE V AND I SOURCES?  We may need to define a `super-mesh' (this is where the current is found common to both meshes) See Figure \ref{fig_supermesh_image}
\end{enumerate}

\begin{figure}
  \includegraphics[width=\textwidth]{/02_UqSubjects/supermesh}
  \caption{Snapshot - Super-Mesh in Circuit Analysis (Mesh)}
  \label{fig_supermesh_image}
\end{figure}

\includegraphics[width=\textwidth]{/02_UqSubjects/kirchhoffs}

\textsc{\large Superposition}\\
Applies to all linear systems:
\begin{itemize}
  \item allows each source in a circuit can be treated independently
  \item and then, complete response of a circuit is calculated by adding the impact of each independent source.
  \item \textcolor{red}{One active source. Set all other sources to zero.}
  \item \textcolor{red}{Replace current source with an open circuit (0 amps flowing)}
  \item \textcolor{red}{Replace voltage source with a short circuit (0 volts across source)}
  \item calculate the voltage/current at point of interest
  \item repeat for each source in the circuits
  \item add the voltage/current components from each source at the point of interest
  \item \textcolor{red}{REMEMBER TO CONSISTENTLY FOLLOW THE SIGN CONVENTION!}
\end{itemize}

\textsc{\large Linear Circuit Theorems}\\
These below theorems are used in practical network calcalations... i.e in ELEC4300 and ELEC4302 for fault current levels; for the Positive, Negative and Zero Sequence equations.
\begin{itemize}
  \item \textbf{Thevenin's theorem} states that any resistive circuit can be replaced by a voltage source and a series resistance; and can be extended to circuits with L and C. It's characterised by: open-circuit voltage $v_{oc}$, short-circuit current $i_{sc}$ and series resistance $R_t$; where $v_{oc} = v_s$ [voltage open circuit is now the thevenin voltage source], $R_t = R_s$ [equivalent resistance is now the thevenin resistance], and $i_{sc} \frac{v_s}{R_s}$ [thevenin current calculated once you have gotten the other two variables].
  \item \textbf{Norton transformation} means that any circuit may be modelled by a current source with a parallel resistance; It's characterised by: short-circuit current $i_{sc}$, open-circuit voltage $v_{oc}$ and parallel resistance $R_p$; where $i_{sc} = i_s$ [current short circuit is now the thevenin current source], $v_{oc} = i_s R_p$ [open-circuit voltage voltage is calculated once you have gotten the other two variables].
\end{itemize}

Thevenin and Norton can be transformed to each other (Figure \ref{fig_theveninSourceTrans}):
\begin{figure}
  \includegraphics[width=\textwidth]{/02_UqSubjects/thevSourceTransformation}
  \caption{Snapshot - Thevenin and Norton Source Transformation}
  \label{fig_theveninSourceTrans}
\end{figure}

Other topics found:
\begin{itemize}
  \item Definitions: Ideal Sources (Voltage, Current), Independent vs Dependent Sources
  \item Approaches to solve circuits
  \item Maximum power transfer
  \item Class exercises
\end{itemize}
\clearpage

\subsubsection{LEC03: Operational Amplifiers (Op Amps)}
Other topics found:
\begin{itemize}
  \item Solving op amps
  \item Inverting, summing, non-inverting amplifier
  \item Voltage follower
  \item Differential amplifier
  \item Virtual earth
  \item Amplifier limitations
  \item Class exercises
  \item MUCH MORE!!!
\end{itemize}

\subsubsection{LEC05: Integrators and Differentiators}
Other topics found:
\begin{itemize}
  \item Control systems engineering - Integrators and Differentiators
  \item etc
\end{itemize}
\clearpage

\subsubsection{LEC06: RLC Circuits}
\textsc{\large Parallel RLC}\\
RLCs means that we have to solve second order differential equations (ODEs). See Figure \ref{fig_parallelRLC}
\begin{figure}
  \includegraphics[width=\textwidth]{/02_UqSubjects/parallelRLC}
  \caption{Snapshot - Parallel RLC Circuit}
  \label{fig_parallelRLC}
\end{figure}

Assume: $v = Ae^{st}$, thus using Laplace transforms to solve the differential equation: $s^2 + \frac{s}{RC} + \frac{1}{LC}$. Solve using the quadratic formula.

Thus, the natural response of parallel RLC:
\begin{align}
  v &= A_1 e^{s_1 t} + A_2 e^{s_2 t} \\
  s_1 &= -\frac{1}{2RC} + \sqrt{(\frac{1}{2RC})^2 - \frac{1}{LC}} \\
  s_2 &= -\frac{1}{2RC} - \sqrt{(\frac{1}{2RC})^2 - \frac{1}{LC}}
\end{align}
We define damping coefficient and undamped natural frequency as:
\begin{align}
  \alpha &= \frac{1}{2RC} \\
  \omega &= \frac{1}{\sqrt{LC}}
\end{align}

Determining the damping, we need to find the roots of the characteristic equation:
\begin{align}
  s_1 &= -\alpha + \sqrt{\alpha^2 - \omega_0^2} \\
  s_2 &= -\alpha - \sqrt{\alpha^2 - \omega_0^2}
\end{align}
Thus, by defining $\zeta = \frac{\alpha}{\omega_0}$ (damping ratio), there are three cases:
\begin{enumerate}
  \item Overdamped case ($\zeta > 1$): $s_1 > s_2$ are both real.
  \item Critically damped case ($\zeta = 1$): $s_1 = s_2 = -\alpha = -\omega_0$.
  \item Underdamped case ($\zeta < 1$): $s_1$ and $s_2$ are complex.
\end{enumerate}

Over-damped case $\alpha^2 > \omega_0^2$:
\begin{align}
  v(t) &= A_1 e^{s_1t} + A_2 e^{s_2t}
\end{align}
$A_1$ and $A_2$ are determined by solving the following:
\begin{align}
  v(0^+) &= A_1 + A_2 \\
  \frac{dv(0^+)}{dt} &= \frac{i_C(0^+)}{C} = s_1 A_1 + s_2 A_2
\end{align}

Critically damped case $\alpha^2 = \omega_0^2$:
\begin{align}
  v(t) &= D_1 t e^{-\alpha t} + D_2 t e^{-\alpha t}
\end{align}
$D_1$ and $D_2$ are determined by solving the following:
\begin{align}
  v(0^+) = D_2
  \frac{dv(0^+)}{dt} &= \frac{i_C(0^+)}{C} = D_1 - \alpha D_2
\end{align}

Under-damped case $\alpha^2 < \omega_0^2$:
\begin{align}
  v(t) &= B_1 e^{-\alpha t} cos (\omega_d t) + B_2 e^{-\alpha t} sin (\omega_d t)
\end{align}
where $\omega_d = \sqrt{\omega_0^2 - \alpha^2}$ denotes damped natural frequency.
$B_1$ and $B_2$ are determined by solving the following:
\begin{align}
  v(0^+) &= B_1 \\
  \frac{dv(0^+)}{dt} &= \frac{i_C(0^+)}{C} = -\alpha_1 B_1 + \omega_d B_2
\end{align}

\textsc{\large Series RLC}\\
\begin{figure}
  \includegraphics[width=\textwidth]{/02_UqSubjects/seriesRLC}
  \caption{Snapshot - Series RLC Circuit}
  \label{fig_seriesRLC}
\end{figure}

Thus, the resulting characteristic equation is $s^2 + \frac{R}{L}s + \frac{1}{LC} = 0$. The roots are:
\begin{align}
  s_{1,2} &= -\frac{R}{2L} \pm \sqrt{(\frac{R}{2L})^2 - \frac{1}{LC}} \\
  s_{1,2} &= -\alpha \pm \sqrt{\alpha^2 - \omega_0^2}
\end{align}

We define damping coefficient and undamped natural frequency as (for Series RLC):
\begin{align}
  \alpha &= \frac{R}{2L} \\
  \omega &= \frac{1}{\sqrt{LC}}
\end{align}

Thus, there are three possible solutions:
\begin{enumerate}
  \item $i(t) = A_1e^{s_1t} + A_2e^{s_2t}$ (over damped)
  \item $i(t) = B_1e^{-\alpha t}cos{\omega_d t} + B_2e^{-\alpha t}sin{\omega_d t}$ (under damped)
  \item $i(t) = D_1 t e^{-\alpha t} + D_2 t e^{-\alpha t}$ (critically damped)
\end{enumerate}

Other topics found:
\begin{itemize}
  \item Class exercises
\end{itemize}

\subsubsection{LEC07: Phasor Analysis}
\textsc{\large Phasors}\\
Up to phasors
\clearpage

\subsection{ELEC3100 - Advanced Electrical Theory}
\clearpage

\subsection{ELEC3300 - Motors \& Electrical Energy}
\clearpage

\subsection{ELEC3400 - Amplifiers \& Electronics}
\clearpage

\subsection{ELEC4300 - Power System Analysis}
\clearpage

\subsection{ELEC4302 - Power System Protection}
\clearpage

\subsection{ELEC4620 - Signal Processing}
\clearpage

\subsection{ELEC4630 - Image Processing}
\clearpage

% TOPIC: PROJECT MANAGEMENT
\subsection{ENGG4800 - Project Management}
\clearpage

% TOPIC: CONTROLS
\subsection{METR4201 - Control System Analysis}
\clearpage





\begin{comment}
\begin{figure}[!htpb]
    \vspace{-3mm}
  \centering
    {
        \includegraphics[width=0.8\textwidth]{figures/asset_overview.png}
    }
    \caption{Sample Asset Management User Interface Display: Asset Overview Screen}
    \label{fig:ui_mockup_overview}
    \vspace{-1mm}
\end{figure}

\begin{figure}[!htpb]
    \vspace{-3mm}
  \centering
  	{
        \includegraphics[width=0.9\textwidth]{figures/asset_specific.png}
    }
    \caption{Sample Asset Management User Interface Display: Asset Lookup Screen}
    \label{fig:ui_mockup_jackHammer}
    \vspace{-1mm}
\end{figure}

Megacorp’s culture is to diversify their knowledge base and keep an open mind for new projects.  The project will align with Megacorp’s by giving employees the opportunity to grow in their technical knowledge, leadership and teamwork. The proposed asset management system will be largely targeted to construction and mining companies. This leverages a customer base, most of whom, may already be Megacorp customers. The project stakeholders include Megacorp and construction company clientele.\\

With more than 70,000 construction companies listed in Australia, the potential market size is 3500 clients if 5\% penetration is assumed~\cite{RefI2}. Table~\ref{tab_competitors} below gives an overview of the features of the proposed product compared to the closest competitors. Several services and features of the proposed product will address deficiencies of the existing competitors and set Megacorp ahead of the competition.

\begin{table}[!htpb]
    \vspace{-1mm}
  \centering
  	{
        \includegraphics[width=0.8\textwidth]{figures/competitors.png}
    }
    \caption{Asset Management Product Competitor Comparison~\cite{RefI3, RefI4, RefI5, RefI6, RefI7} (Key: green = product capability, yellow = product deficiency)}
    \label{tab_competitors}
    \vspace{-1mm}
\end{table}
\end{comment}
